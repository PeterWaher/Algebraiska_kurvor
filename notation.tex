\chapter{Notation}

De symboler och notation som använts i detta dokument har kategoriserats och listats här nedan.

\section{Mängder}

\begin{tabular}{cp{10.5cm}}
$\varnothing$	& Den tomma mängden \\ 
$\in$	& \ldots är ett element i \ldots \\ 
$\notin$	&  \ldots är inte ett element i \ldots \\ 
$\cup$	& Unionen av två mängder \\ 
$\cap$	&  Skärningen av två mängder \\ 
$\setminus$	&  Komplementet av den högra mängden i den vänstra \\ 
$\subset$	& \ldots är en strikt delmängd av \ldots, dvs. de är inte lika \\ 
$\subseteq$	& \ldots är en delmängd av \ldots, och de kan vara lika \\ 
$\{ \ldots \}$	& Beskrivning av en mängd \\ 
$\left\|  \ldots \right\|$	& Antalet element i mängden \\ 
$\left\langle \ldots \right\rangle $	& Den mängd, semigrupp, grupp, ring, modul eller algebra som genereras av \ldots, beroende av kontexten. \\ 
\end{tabular} 

\section{Fördefinierade mängder}

\begin{tabular}{lp{10.5cm}}
$\mathbb{N}$ & De naturliga talen, inklusive noll: $0, 1, 2, 3, \ldots$ \\
$\mathbb{Z^+}$ & De positiva heltalen (exklusive noll): $1, 2, 3, \ldots$ \\
$\mathbb{Z}$ & Alla heltal \\
$\mathbb{R}$ & De reella talen \\
$\mathbb{C}$ & De komplexa talen
\end{tabular}

\section{Heltal}

\begin{tabular}{lp{10.5cm}}
$\mathbb{Z}_m$ & $\left\lbrace \left[ n \right] _m : n \in \mathbb{Z}\right\rbrace$ \,\, (skrivs även $\mathbb{Z}/\left\langle m \right\rangle$, $\mathbb{Z}/m$ eller $\mathbb{Z}/{m\mathbb{Z}}$) \\
$\left[ n \right] _m$ & $\left\lbrace a \in \mathbb{Z} : a \equiv n \pmod{m}\right\rbrace$ \\
$a \equiv n \pmod{m}$ & $\exists b \in \mathbb{Z} : a = n + b \cdot m$ \\
$a \mid b$ & $a$ delar $b$, dvs. $\exists n \in \mathbb{N} : b = a \cdot n$ \\
$a \nmid b$ & $a$ delar inte $b$
\end{tabular}

\section{Logiska operatorer}

\begin{tabular}{cp{10.5cm}}
$\exists$ & Existens av \ldots \\
$\forall$ & För alla \ldots \\
$\Longrightarrow$ & Logisk implikation \\
$\Longleftrightarrow$ & Logisk ekvivalens \\
$\vee$ & Logiskt eller \\
$\wedge$ & Logiskt och \\
$\neg$ & Logisk negation \\
$\therefore$ & Därför, logisk slutsats
\end{tabular}

\section{Relationer}

\begin{tabular}{cp{10.5cm}}
$\approx$ & $f \approx g$ betyder att $f$ är ungefär lika med $g$. Kan vara ett empiriskt antagande eller en förenkling.\\
$\sim$ & $f \sim g$ (då $n \rightarrow \infty$) betyder att $f$ är asymptotisk till $g$ (då $n$ går mot $\infty$), dvs. $\lim\limits_{n \rightarrow \infty}\frac{f(n)}{g(n)}=1$.\\
\end{tabular}
\section{Övrigt}

\begin{tabular}{cp{10.5cm}}
$0$ & Nollelementet i gruppen som behandlas. Kan även stå som förkortning
av $\left\lbrace 0\right\rbrace$. \\
$1$ & Ett elementet i ringen som behandlas. Kan även stå som förkortning
av $\left\lbrace 1\right\rbrace$. \\
$\mathbf{0}$ & $\left(0, \ldots , 0\right) \in \mathbb{C} ^n$ \\
$\infty$ & Oändligheten. Vilken oändlighet framgår av sammanhanget. \\
\end{tabular}
